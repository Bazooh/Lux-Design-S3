\documentclass[12pt,a4paper]{article}

% Packages utiles
\usepackage[utf8]{inputenc}
\usepackage[T1]{fontenc}
\usepackage{geometry}
\geometry{margin=1in}
\usepackage{graphicx}
\usepackage{amsmath}
\usepackage{hyperref}

% Titre du document
\title{Rapport de projet Deep Learning
\\ \large Compétition Kaggle - NeurIPS 2024 - Lux AI Season 3
}
\author{Pierre Jourdin \\ Aymeric Conti \\ Hadrien Crassous}
\date{Pour le 2 mars 2025}

\begin{document}

\maketitle

\tableofcontents

https://www.kaggle.com/competitions/lux-ai-season-3/overview

au début, on a essayé de faire du single agent, pour avoir une base simple qui marche, mais en fait c'est pas simple car naivement l'action space est de taille 5^16.
on essaie du MARL direct du coup.

\end{document}
